\documentclass[times, 10pt,twocolumn]{article} 
\usepackage{latex8}    
\usepackage{times}
\usepackage{url}  
\usepackage{listings}
\newcounter{firstbib}
 
\pagestyle{empty}

\begin{document} 

\title{ Large Scale and High Performance NoSQL Databases in Web Applications}
\author{Mateusz Bilski, mateusz.bilski@gmail.com \\ Ireneusz Kawalec, 180528@student.pwr.wroc.pl \\ \\
Wroclaw University of Technology\\ Faculty of Electronics \\ Computer Science \\ Internet Engineering  \\  
} 

\maketitle
\thispagestyle{empty} 

\begin{abstract}  

The aim of this document is to present comparision between available NoSQL databases and architecture which
provides high performance for large scale web applications.

Most known 10 NoSQL databases has been investigated and compared with each other. We have analyzed the advantages 
and disadvantages of using NoSQL databases in compare with SQL databases. 

We choosed the best NoSQL database and built a system which provide fast response time on reading and writing operation
under heavy load when contains huge size of data.

As the outcome of this paper we described the architecture, configuration and presented results of tests.

\end{abstract} 

keywords: database, NoSQL, large scale, high performance, web application

%------------------------------------------------------------------------- 
\Section{Introduction}

Databases are used almost in all aspects of software engineering. Each 
information system needs a place to storage a data. However, due to the complexity of the systems 
the key role is to provide the proper solution.

Modern web applications should handle a huge load and maintain high performance and
availability. Good examples of such services are Facebook, Amazon, eBay and Google Mail.

Relational databases occur to be inefficient in this case. Strong complexity and
heave coupling causes low performance. The solution for that issues are Object Oriented
Databases which ensure high scalability.

Our aim is to compare available NoSQL databases and to select the best one by evaluating 
optimal architecture that will fulfill requirements. 

Database is organized collection of data. It can be discriminated two types, 
Relation Databases and Object Oriented Databases.

Relation database consists of specific number of tables and each has a distinguish column named primary key. 
They are matured, well documented, tested and commonly used. Several examples of then are: MySQL, Oracle, MS SQL, PostgreSQL. 

Object oriented database was developed in response to expansion of object oriented programming. In addition to ensure high scalability and  
performance in current information systems.

The architecture of database system has a big influence of performance. Single machine has a limited
resources, so this is a reason why enterprise systems  are composed of a lot of them.

%------------------------------------------------------------------------- 
\Section{Research Questions}

In our paper we want to give answers to the following questions:

\begin{itemize}
  \item What are all available NoSQL databases
  \item What are the differences between building database system based on relation and object model
  \item Which NoSQL database is the best for large scale, high performance database systems
  \item How configure database to gain maximum efficency
  \item How to carry out database performance testing
\end{itemize}

The expected outcomes of the research are:

\begin{itemize}
  \item comparision of most known NoSQL databases
  \item comparision beetween NoSQL and SQL databases
  \item architecture and configuration of chosen NoSQL database which provides high performance on large scale database systems
  \item tests results
\end{itemize}
 
\Section{Research method}

\Section{Analysis}    

\Section{Synthesize}   

\Section{Conclusion} 

\begin{thebibliography}{9}  

\setcounter{enumiv}{\value{firstbib}}

 \bibitem{strauch}
  	  Christof Strauch, 
      \emph{NoSQL Databases}. Stuttgart Media University.
      
      \bibitem{survey}
  	  Jing Han, Haihong E, Guan Le,
      \emph{Survey on NoSQL Database}. Beijing University of Posts and Telecommunications.
      
      \bibitem{strauch2}
  	  Christof Strauch, 
      \emph{NoSQL Databases: a step to database scalability in Web environment}. Stuttgart Media University.
      
      \bibitem{pokorny}
  	   Jaroslav Pokorny, 
      \emph{NoSQL Databases}. Charles University.
      
      \bibitem{cattell}
  	  Rick Cattell, 
      \emph{Scalable SQL and NoSQL Data Stores}. Cattell.Net Software.
      
      \bibitem{leavittm}
  	  Neal Leavittm 
      \emph{Will NoSQL Databases Live Up to Their Promise}. Technology News.
      
      \bibitem{delis}
  	  Alexios Delis, Nick Roussopoulos, 
      \emph{Performance and Scalability of Client-Server Database Architectures}. University of Maryland.
      
      \bibitem{tudorica}
  	  Bogdan George Tudorica, Cristian Bucur, 
      \emph{A comparison between several NoSQL databases with comments and notes}. Petroleum-Gas University of Ploiesti, Ploiesti, Romania.
      
      \bibitem{taniar}
  	  David Taniar, 
      \emph{High Performance Database Processing}. Clayton School of Information Technology, Clayton, Victoria, Australia.
      
      \bibitem{konstantinou}
  	  Ioannis Konstantinou, Evangelos Angelou, Christina Boumpouka, Dimitrios Tsoumakos, Nectarios Koziris, 
      \emph{On the Elasticity of NoSQL Databases over Cloud Management Platforms}. National Technical University of Athens, Greece.
      
      \bibitem{chakrabarti}
  	  Aniket Chakrabarti, Christopher Stewart,
      \emph{Efficent Latency Management in NoSQL Stores}. The Ohio State University.
      
      \bibitem{rizzotti}
  	  Nicolas Ruflin, Helmar Burkhart, Sven Rizzotti, 
      \emph{Social-Data Storage-Systems}.  University of Basel, Switzerland.
      
      \bibitem{tiwari}
  	  Shashank Tiwari, 
      \emph{Professional NoSQL}. Indianapolis, Ind. : John Wiley \& Sons, 2011.
      
      \bibitem{orend}
  	  Kai Orend, 
      \emph{Analysis and Classification of NoSQL Databases and Evaluation of their Ability to Replace an Object-relational Persistence Layer}. Master's Thesis. Technische Universitat Munchen.
      
      \bibitem{prasad}
  	  Rabi Prasad, 
      \emph{RDBMS to NoSQL: Reviewing Some Next-Generation Non-Relational Database's}. Stuttgart Media University.
      
      \bibitem{hecht}
  	  Robin Hecht, Stefan Jablonski,
      \emph{NoSQL Evaluation - A Use Case Oriented Survey}. Patra Berhampur University India.
       
      \bibitem{sakr} 
  	  Sherif Sakr, 
      \emph{Supply cloud-level data scalability with NoSQL databases}. National ICT Australia.
      
      \bibitem{nosql-database}
      \emph{The biggest list of nosql databases} available at
      \url{http://nosql-database.org/}

\end{thebibliography}

\end{document}
