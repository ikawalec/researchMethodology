\documentclass[gray]{beamer}
%\setbeamertemplate{headline}{\scriptsize{\vspace*{0.3cm}\hspace*{0.3cm}\insertframenumber}} 

\setbeamertemplate{footline}[page number]{}

%gets rid of navigation symbols
\setbeamertemplate{navigation symbols}{}

\begin{document}
\title{Large Scale and High Performance NoSQL Databases in Web Applications \\ Project Proposal}  
 
\author{Mateusz Bilski, mateusz.bilski@gmail.com \\ Ireneusz Kawalec, ireneusz.kawalec@gmail.com} 
\date{\today} 
\institute[PWR]{Wroclaw University of Technology\\ Faculty of Electronics \\ Computer Science \\ Internet Engineering}

 
\frame{\titlepage}     
 
\frame{\frametitle{Plan of presentation}\tableofcontents} 
   
\section{Introduction}   

\frame{\frametitle{What is database?}    

A database is an organized collection of data. It is designed to offer an organized mechanism for storing, 
managing and retrieving information. 

\vspace{1cm}

Databases can be classified according to their organizational approach. Two main types can be distingeshed: 
\begin{itemize}
  \item relational database
  \item no-relational database  
\end{itemize} 

}

\frame{\frametitle{Relational database}

In Relational Database:
\begin{itemize}
  \item information is represented as a collection of tables
  \item the table consist of columns and rows
  \item each column contains may have a different type of attribute
  \item one attribute in table can be distingueshed as primary key
  \item foreign keys can be used to create relations between tables 
\end{itemize}

% TODO img 

}
%\frame{\frametitle{Object database}

%In Object Database information is represented in the form of objects as used in object-oriented programming.
%It extends relational data model: Simple and complex objects, attributes, methods, classes, inheritance, references.
% TODO img

%}


%\frame{\frametitle{The problem}

% Relational databases are not efficient in system which huge load. 

% problemem jest to że relacyjne bazy danych nie sprawdzają się w systemach gdzie wprowadza się dużo danych

%}

\frame{\frametitle{No-relational database} 

A NoSQL database provides a mechanism for storage and retrieval of data that use looser consistency models than 
traditional relational databases in order to achieve horizontal scaling and higher availability. 

%Some authors refer to them as "Not only SQL" to emphasize that some NoSQL systems do allow SQL-like query language to be used.

\vspace{1cm}

Examples:
\begin{itemize}
  \item Facebook - Casandra
  \item Google - Big Table
  \item Amazon - SimpleDB 
\end{itemize}
}

\section{Outcomes of Systematic Literature Review}  
\frame{\frametitle{Outcomes of Systematic Literature Review} 

We have done Systematic Literature View for Large Scale and High Performance NoSQL Databases in Web Applications.

\vspace{1cm}

Outcomes:
\begin{itemize}
  \item some studies describing most popular NoSQL databases, i.e. MongoDB, Casandra, BigTable
  \item no document describing architecture, configuration and tests results of usage NoSQL databases for
large scale systems which requires high performance
\end{itemize}

}
 
\section{Aims and Objectives}  
\frame{\frametitle{Aims and Objectives} 

Aims
\begin{itemize}
	\item to evaluate performance of NoSQL databases for large scale web applications.
\end{itemize}

\vspace{1cm}

Objectives
\begin{itemize}
	\item to design architecture of system for high performance and large scale web application using NoSQL databases
	\item to develop an optimal configuration for prepared architecture for each database
	\item to measure reading and writing response time of the system for each database
\end{itemize} 

}

\section{Expected outcomes}  
\frame{\frametitle{Expected outcomes} 

\begin{itemize}
  \item architecture and configuration of NoSQL databases which provides high performance on large scale database systems
  \item tests results of measuring reading/writing response time of each NoSQL databases on prepared architecture and configuration.
\end{itemize}

}

\section{Research questions and methods}  
\frame{\frametitle{Research questions and methods} 

Research questions
\begin{itemize}
  \item What is the best architecture for large scale, high performance NoSQL databases?
  \item How to configure NoSQL databases to gain maximum efficiency?
  \item How to carry out database performance testing?
  \item Which NoSQL database is the best for large scale, high performance database systems?
\end{itemize}


Methods
\begin{itemize}
  \item Action Research - for finding out the best architecute of system and configuration of each database
  \item Experiment - for measuring reading/writing response times of each NoSQL databases on prepared architecture and configuration.
\end{itemize}

}

\section{Project plan}  
\frame{\frametitle{Project plan} 

\begin{enumerate}
  \item Choose the five most known NoSQL databases
	% 1.1 przejrzenie literatury pod wzledem baz danych NoSQL
	% 1.2 wybranie 5ciu baz najczesciej wystepujacych w wynikach wyszukiwania 
  \item Design an architecture of the system
    % 2.1 jakie metody stosuje sie przy projektowaniu duzych systemow bazodanowych
    % 2.2 ktore z nich zapewniaja high performance
  \item Find out the configuration for each database to be the most efficient on prepared architecture
    % tuningowanie względem maszyny (ram, dysk, procek, sieciówka) 
    % tuningowanie względem danych 
  \item Measure response times of each database
    \begin{enumerate}
      \item read time  
      \item write time 
    \end{enumerate}  
\end{enumerate}

}

\section{References}
\frame{\frametitle{References}

\begin{thebibliography}{9}
\bibitem{asd}
Ludwik Kuzniarz
\emph{Lecture 5: Writing Project Proposal, Research Methodology course at WUT}. Blekinge Institute of Technology
 
\bibitem{qwe}  
Mateusz Bilski, Ireneusz Kawalec
\emph{Systematic Literature Review of Large Scale and High Performance NoSQL Databases in Web Applications}. Wroclaw University of Technology
  
  
% TODO dostęp online  
\bibitem{zxc}  
\emph{List of NoSQL Databases} \\
\url{http://nosql-database.org/}
\end{thebibliography}

}

\end{document}
