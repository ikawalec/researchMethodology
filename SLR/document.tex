\documentclass[times, 10pt,twocolumn]{article} 
\usepackage{latex8}    
\usepackage{times}  

\pagestyle{empty}

\begin{document}

\title{ Systematic Literature Review
       for Large Scale and High Performance NoSQL Databases in Web Applications}

\author{Mateusz Bilski\\ Ireneusz Kawalec \\ Radoslaw Wilczak\\ \\
Wroclaw University of Technology\\ Faculty of Electronics \\ Computer Science
}

\maketitle
\thispagestyle{empty}

\begin{abstract}  

The reason of this document is to normalise, support and define Systematic Literature Review process for the topic.
It would be define problems that lurk for complex database systems used in big enterprises. 
It would be discuss the differences, virtue and vice of database systems based on Relational Databases and Object Orientated Databases.
It would be list objectives which will help achieving the aim.

\end{abstract} 



%------------------------------------------------------------------------- 
\Section{Introduction}

Databases are used almost in all aspects of software engineering. Each 
information systems needs a place to storage a data. But each system is different
and requires different approaches. The key roles is to select the right solution.
In nowadays global web applications involves handling of huge load, high performance and
availability. Good examples of such services are Facebook or Google Mail.

Relational databases occurs to be inefficient in this case. Strong complexity and
heave coupling causes low performance. The solution for that issues are Object Oriented
Databases which ensure high scalability.

Our aim is to compare available NoSQL databases and to select the best one by evaluating 
optimal architecture that will fulfil requirements. 

The first step to achieve this target is to investigate systematic literature review in this area.
By doing it we want to find out answers for our research questions.

This kind of knowledge is really valuable from enterprise perspective. However each database
system has different requirements so it is impossible to prepare one common solution.
 
%------------------------------------------------------------------------- 
\Section{Background}

Database is organized collection of data. It can be discriminated two types, 
Relation Databases and Object Oriented Databases.

Relation database consists of specific number of tables and each has a distinguish column named primary key. 
They are matured, well documented, tested and commonly used. Several examples of then are: MySQL, Oracle, MS SQL, PostgreSQL. 

Object oriented database was developed in response to expansion of object oriented programming. In addition to ensure high scalability and  
performance in current information systems.

The architecture of database system has a big influence of performance. Single machine has a limited
resources, so this is a reason why enterprise systems  are composed of a lot of them.


%------------------------------------------------------------------------- 
\Section{Research Questions}

During the Systematic Research Review following questions will be posed: 

maxymalnie dwa punkty, wywalic

\begin{itemize}
  \item What are all available NoSQL databases
  \item co do tej pory zostalo zreserczerowane na ten temat 
\end{itemize}

tutaj outcomes - co jest wynikiem researchu - spodziewane

\Section{Research method}

TODO (library, google scholar)
tutaj streszczamy co to jest SLR na podstawie kiczermana. Ze taka wybralismy
co to jest, jak to wyglada itp

%------------------------------------------------------------------------- 
\Section{Protocol}

opisujemy jak wyglada proces naszego litricze rewiew.

\SubSection{Inclusion and exclusion criteria}

czyli jakich tematow szukamy w materia?ach
i ktore tematy odrzucamy
czyli np. znalezlismy jakis material rozwodzacy sie nad zaletami i wadami jakies malo istotnej cechy jakies implementacji konkretnej bazy NoSQL = TO NAS NIE INTERESUJE, robimy exclusion. 
Papiery ktory byly malo profesjonalne, na ktorych ktos sie uczyl, itp. odrzucamy.

\SubSection{Data sources and search strategy}

gdzie szukamy itp

For search strategy it was choose to search digital libraries like:
\begin{itemize}
	\item ACM
	\item IEEEXplore
	\item Google Scholar
\end{itemize}

There were searched using following searched phrase:
\begin{enumerate}
	\item 
	\item
	\item
\end{enumerate}


%------------------------------------------------------------------------- 
\Section{Execution of protocol}

opisac jak to sie tam szukalo, czyli ze szukalismy, znalezlismy 200,
wyjebalismy 100 na podstawie czegos, potem wyjebalismy 80 na podstawie 
czegos tam innego. A potem po przeczytaniu ich wyjebalismy jescze 16 i zostaly
4

\ldots

%------------------------------------------------------------------------ 

\Section{Analisis}   

na podstawie protokolu, tych prac ktore wybralismy, czytami i dokonujemy analizy ich.
analiza to badanie kazdej pracy.

\Section{Synthesize}   

A synteza to przetowozenie wynikow i zabranie tego do kupy.

\ldots

%------------------------------------------------------------------------- 
\Section{Conclusion} 

trzeba napisac ze takie a takie rzeczy sa juz opisane dobrze, ale znalezlismy dziura, ktora
nie zostala jeszcze dobrze zliteralizowana i sie fajnie

%------------------------------------------------------------------------- 
\Section{References}
\noindent [1] B.A. Kitchenham, Guidelines for performing Systematic Literature Reviews in Software Engineering Version 2.3, Keele University and University of Durham, EBSE Technical Report, 2007. \newline
[2] \ldots \newline
[3] \ldots

%------------------------------------------------------------------------- 
\nocite{ex1,ex2}
\bibliographystyle{latex8} 

HARMONOGRAM
Radek: Reseach Method : poniedziałek 8 rano
Mateusz: Protocol : wtorek 8 rano
Ireneusz: Execution of protocol czwartek 8 rano
Radek: Analisis sobota 8 rano
Mateusz: Conclusion sobota 8 rano
Irek: Synthesiza sobota 8 rano
Niedziela wieczor: spotykamy sie i dopinamy calos + initial assigmenet 2 o 17

\end{document}
