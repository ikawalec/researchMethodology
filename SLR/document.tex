\documentclass[times, 10pt,twocolumn]{article} 
\usepackage{latex8}    
\usepackage{times}  

\pagestyle{empty}

\begin{document}

\title{ Systematic Literature Review
       for Large Scale and High Performance NoSQL Databases in Web Applications}

\author{Mateusz Bilski\\ Ireneusz Kawalec \\ Radoslaw Wilczak\\
Wroclaw University of Technology\\ Faculty of Electronics \\ Computer Science
}

\maketitle
\thispagestyle{empty}

\begin{abstract}

The reason of this document is to normalise, support and define Systematic Literature Review process. 
It would be define problems that lurk for complex database systems used in big enterprises. 
It would be discuss the differences, virtue and vice of database systems based on Relational Databases and Object Orientated Databases.
It would be list objectives which will help achieving the aim.

\end{abstract}



%------------------------------------------------------------------------- 
\Section{Introduction}

Databases are used almost in all aspects of software engineering. 
Each information systems needs a place to storage a data. But each system is different
a requires different approaches. The key roles is to select the right solution.
In nowadays global web applications involves handling of huge load, high performance and availability. Good examples of such services are Facebook or Google Mail.

Relational databases occurs to be inefficient in this case. Strong complexity and heave coupling causes low performance.



% Z pomoca przychodza obiektowe bazy danych ktore gwarantuja wysoka skalowalnosc

Our aim is to compare available NoSQL databases and to select the best one by evaluating 
optimal architecture that will fulfil reqirements. 

The first step to achieve this target is to investigate systematic literature review in this area. 
% robiac systematic literature review chcemy sie dowidzedizec do do tej pory wiadome jest
% na temat relacyjnych baz danych i my chcemy zrobic research jak zrobic obiektowe 


% to jest wiedza firm ktore nie chca sie dzielic.

%------------------------------------------------------------------------- 
\Section{Background}

Database is organised collection of data. It can be discriminated two types, 
Relation Databases and Object Oriented Databases.

Relation database consists of specific number of tables and each has a distinguish column named primary key. 
They are matured, well documented, tested and commonly used. Several examples of then are: MySQL, Oracle, MS SQL, PostgreSQL. 

Object oriented database was developed in response to expansion of object oriented programming. In addition to ensure high scalability and  
performance in current information systems\ldots

A lot of 


%------------------------------------------------------------------------- 
\Section{Research Questions}
Chcemy postawić następujące pytania:

1. chcemy wiedzieć jaka baza danych NoSQL jest najlepsza dla large scale, high performance
jaka architektura jest najlepsza

2. jak robić testy wydajnościowe baz danych

3. jak tworzyć high performence database systems /architecture

4. jakie są dostępny WSZYSTKIE NoSQL databases

5. jakie są różnice w budowaniu systmów bazodanowyego w opraciu o relacyjny model bazodanowy a obiektowy


\Section{Research method}

go to library
google it

%------------------------------------------------------------------------- 
\Section{Protocol}

TODO

%------------------------------------------------------------------------- 
\Section{Execution of protocol}

TODO


%------------------------------------------------------------------------ 
\Section{Synthesize}

TODO

%------------------------------------------------------------------------- 
\Section{Conclusion}

Robim co mozem

%------------------------------------------------------------------------- 
\Section{References}
[1] Kitcheman
[2] TODO
[3] TODO


%------------------------------------------------------------------------- 
\nocite{ex1,ex2}
\bibliographystyle{latex8}
\bibliography{latex8}

\end{document}

