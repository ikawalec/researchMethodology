% !TEX encoding = UTF-8 Unicode
\documentclass[times, 10pt,twocolumn]{article} 
\usepackage{latex8}    
\usepackage{times}
\usepackage{url}  
\usepackage{listings}
\newcounter{firstbib}
 
\pagestyle{empty}

\begin{document}  

\title{ Systematic Literature Review
       of Large Scale and High Performance NoSQL Databases in Web Applications}
\author{Mateusz Bilski, mateusz.bilski@gmail.com \\ Ireneusz Kawalec, ireneusz.kawalec@gmail.com \\ 
Wroclaw University of Technology\\ Faculty of Electronics \\ Computer Science \\ Internet Engineering  \\  
}

\maketitle
\thispagestyle{empty} 

\begin{abstract}  

The aim of this document is to define and describe results of Systematic Literature Review process which 
refers to the area of Large Scale and High Performance NoSQL Databases in Web Applications.

This document contains a protocol describig the way of finding crucial informations related to complex database systems used in the big enterprises.   
Both Relational Databases and Object Orientated Databases were concerned.

The execution of protocol determines repetitive process in which such elements as  
inclusion and exclusion criterias, researched data sources were described. 

The concluions delivers the answers to the following questions:
What NoSQL databases are described in the publications? What is known about high performance and large scale NoSQL databases?

\end{abstract} 

keywords: database, NoSQL, large scale, high performance, web application

%------------------------------------------------------------------------- 
\Section{Introduction}

Databases are used almost in all aspects of software engineering. Each 
information system needs a place to storage a data. However, due to the complexity of the systems 
the key role is to provide the proper solution.

Modern web applications should handle a huge load and maintain high performance and
availability. Good examples of such services are Facebook, Amazon, eBay and Google Mail.

Relational databases occur to be inefficient in this case. Strong complexity and
heave coupling causes low performance. The solution for that issues are Object Oriented
Databases which ensure high scalability.

Our aim is to compare available NoSQL databases and to select the best one by evaluating 
optimal architecture that will fulfill requirements. 

The first step to achieve this target is to investigate systematic literature review in this area.
It can be done by finding answers to research questions.

This kind of knowledge is really valuable from enterprise perspective. However, each database
system has different requirements so it is impossible to prepare one common solution. 
 
%------------------------------------------------------------------------- 
\Section{Background}

Database is organised collection of data. It can be categorised into two types: 
Relational databases and Object Oriented Databases.

Relational database consists of specific number of tables and each has a distinguish column named primary key. 
They are well-documented, tested and commonly used. Several examples of then are MySQL, Oracle, MS SQL, PostgreSQL. 
% TODO
% ROZWINAC - OPISAC CZYM SIE ROZNIA, ZALOZENIA COSTAM COSTAM 

Object oriented database was developed in response to expansion of object oriented programming, in addition to ensure high scalability and  
performance in current information systems.

The architecture of database system has a big influence on performance. Single machine has limited
resources, so this is the reason why enterprise systems are composed of multiple boxes named grids.

%------------------------------------------------------------------------- 
\Section{Research Questions}

During the Systematic Literature Review following questions will be posed: 

\begin{itemize}
  \item What NoSQL databases are described in the publications
  \item What is known about high performance and large scale NoSQL databases   
\end{itemize}

% tutaj outcomes - co jest spdziewanym wynikiem researchu - spodziewane
% TODO 
The expected outcome of the research is a collection of publications concerned with the topic of our intereset.
 
\Section{Research method}

The method we have chosen for our research is Systematic Literature Review defined by Kitchenham. For detailed information, please see \cite{kitchenham}. 
~\\
~\\
\emph {A systematic literature review is a means of identifying, evaluating and interpreting all available research relevant to a particular 
research question, or topic area, or phenomenon of interest.} \cite[p.~3]{kitchenham}
~\\
~\\
Systematic Literature Review is a search of all available studies and scientific articles in the field to see what was written on the topic. 
It is done before research work, because it is crucial not to investigate something that is already well explored by someone. To achieve this 
we have to be very honest and rigorous for ourselves and our review. The base of SLR is to create a protocol and present documention. One of 
the purpose of documenting it is to enable to asses the quality of our work. 

We can divide reseach process into 3 stages: planing, conducting and reporting. Firstly we define our needs, then questions, next create and evaluate 
the protocol. In the conducting phase we execute our protocol, estimate the quality of results, extract data out and finally synthesise this 
data. The last stage, reporting review consists of defining how we would distibute effects of our work, composing general report from whole 
SLR and rating it. As can be seen in every stage there is a quality assessment of particular works. The aim is to determine the quality level of final outcomes.
~\\
It is worth to mention that to achieve better effects we should made iterations of individual step or group of steps. During these iterations the protocol should be impoved.

%------------------------------------------------------------------------- 
\Section{Protocol}

We developed a protocol for the systematic review by following the guidelines enclosed in [1]. The protocol specified the research questions, search
strategy, inclusion, exclusion and quality criteria, data extraction, and methods of synthesis.

\SubSection{Inclusion and exclusion criteria}

Studies were eligible for inclusion in the review if the topic was about NoSQL databases. We decided that we needed to find as many information regarding this topic as possible.
The second criteria was if the studies are related to high performance or high scale for either SQL or NoSQL databases. Only studies written in English were included.
We also included papers which cover comparision of available NoSQL databases.

Studies were excluded if they main focus was on SQL databases without considering high performance or large scale problems. 

% to była uwaga: na jakiej podstawie chcemy określać czy system jest complicated?
%In addition if research focused on complicated databases system
%with huge complexity (those which follows third normal form or higher) it was also excluded. 

Furthermore, our research topic is highly related to industry, thus we excluded all papers which does not have any usage in industry.

\SubSection{Data sources and search strategy}  

For the search strategy following digital libraries were examined:
\begin{itemize}
	\item ACM
	\item IEEEXplore
	%\item Compendex
	\item Google Scholar
	%\item SpringerLink
\end{itemize}

Moreover, Google was used to expand result of systematic research review.

The following phrases were used to research data sources:
\begin{enumerate}
    \item NoSQL database 
	\item NoSQL and SQL
	\item high performance database
	\item large scale database
	\item NoSQL comparision
	\item database scalability
\end{enumerate}

%------------------------------------------------------------------------- 
\Section{Execution of protocol}

%opisac jak to sie tam szukalo, czyli ze szukalismy, znalezlismy 200,
%wyjebalismy 100 na podstawie czegos, potem wyjebalismy 80 na podstawie 
%czegos tam innego. A potem po przeczytaniu ich wyjebalismy jescze 16 i zostaly 4

Once we defined the search citeria of our interest, we started literature extration process.
Search phrases were combined by using boolean operators OR and AND:

\emph{(nosql OR sql OR nosql sql) AND database AND (comparison OR high performance OR large scale OR scalability)}
  
We investigated data sources to find answer to research questions.

% w pierwszej interacji otrzymaliśmy
After the first iteration we received 159 materials. In ACM digital library we discovered 70 related results, 41 articles were found in IEEEXplore and finally 48 materials in Google Scholar.

Based on exclusion criteria some of references were rejected. A lot of examinated works were too basic or were focues only on SQL databases.

% TODO
% bardziej rozwinąć na jakiej podstawie wywalaliśmy
% proces iteracyjny 

%Finally we received 16 scientific artices, 1 book, 1 Master Thesis and 1 website which will be described in next section. 
Finally we received 4 scientific artices, 1 book, 1 Master Thesis and 1 website which will be described in next section.

%------------------------------------------------------------------------ 
 
\Section{Analysis}    

%\cite{strauch} w pierwszych rozdziałach swojej pracy opisuje początki powstania baz NoSQL oraz ludzki którzy
%przyczynili się do ich rozwoju. Duży postęp nastąpił podczas rozwoju tzn. Web 2.0 (portale społecznościowe Facebook), serwisy
%internetowe (Amazon, eBay) potrzebowały szybkich rozwiązań mogących przetwarzać dużo danych.

%Autor opisuje zalety baz NoSQL. Należą do ich min. Avoidance of Unneeded Complexity, High Throughput,
%Horizontal Scalability and Running on Commodity Hardware (Machines can be added and removed without causing the same operational efforts to perform sharding in RDBMS cluster-solutions;),
%Movements in Programming Languages and Development Frameworks, High Availability.

%Równocześnie autor podkreśla że są także minusy baz NoSQL. Większość z dostępnych baz to otwarte oprogramowanie, biznesowi 
%ludzie mogą mieć pewne obawy gdy coś nie bedzie działać (nie ma kogo obwiniać). 
%NoSQL to nic nowego, wcześniej były juz próby stworzenia obiektowej bazy danych.

%Autor prezentuje w dziale Performance vs. Scalability obszerne porównianie między bazami SQL i NoSQL.  

%W daleszej części artykułu dokonuje klasyfikacji najpopularniejszych daz NoSQL według 
%dwóch kategori Document Databases (Apache CouchDB, MongoDB) i Column-Orinted Databases (Google’s Bigtable, Cassandra). 

%\cite{tudorica}

% TODO pl na podstawie protokolu, tych prac ktore wybralismy, czytami i dokonujemy analizy ich.
% analiza to badanie kazdej pracy.


Some basic concepts of NoSQL databases, can be found in \cite{survey}, \cite{strauch}, \cite{leavittm} and \cite{prasad}. Interesting 
comparison of NoSQL and SQL databases was promised in article \cite{cattell}. Although a lot of references points to some popular NoSQL database currently 
available and commonly used, but the biggest source was found on website \cite{nosql-database} (almost 150 NoSQL databases). Practial aspects of using NoSQL 
databases in web applications were contained in \cite{tiwari} \cite{orend}, \cite{pokorny}. High performance issues were mentioned in \cite{taniar} and \cite{delis}.
Significant number of the articles refers to cloud solutions (\cite{sakr}, \cite{konstantinou}, \cite{delis}). Some of the articles describes database system with enormous 
complexity \cite{rizzotti}. 

\Section{Synthesize}   

% TODO PL A synteza to przetowozenie wynikow i zabranie tego do kupy.

% Jeśli chodzi o listę dostępnych baz NoSQL to mimo ich dużej ilości tylko 5-7 głownych baz jest dobrze opisane w literaturze.
% Jeśli chodzi o wydajność to dużo odnosi się do usług w chmurze
% Mało jest artykułów które konkretnie opisują jakiś scenariusz i zalejacą którą bazę wybrać do tworzonego projektu

In analysed articles only most popular NoSQL databases were described, e.g HBase, MongoDB, Cassandra, SimpleDB.
Invesitaged works contain valuable information concerning high performance and scalability. 



%Directory of NoSQL databases with basic information on the individual datastores:
%http://nosql-database.org/



%------------------------------------------------------------------------- 
\Section{Conclusion} 

% TODO PL trzeba napisac ze takie a takie rzeczy sa juz opisane dobrze, ale znalezlismy dziura, ktora
% nie zostala jeszcze dobrze zliteralizowana i sie fajnie

% Luką jest brak konkretnych przykładów użycia nosql baz danych w aplikacjach webowych 
Systematic literature review helped us to get to know what NoSQL databases are available 
and what do we know known about their performance and scalability.   

Although there are some solutions describing high performance and scalable NoSQL databases, but there  
is really no specific document which would describe real world scenario like database (type, configuration) for web 
application able to handle 1 milion of users.
% TODO bardziej rozwinąć zdanie powyżej, bo nagle 

That is a luck in knowledge, which could be fulfilled in further research.

%-------------------------------------------------------------------------
% TODO split bibliography 

\renewcommand{\refname}{Article references}

\begin{thebibliography}{9}

	\bibitem{kitchenham} 
	  B.A. Kitchenham,
	  \emph{Guidelines for performing Systematic Literature Reviews in Software Engineering Version 2.3}.
	  Keele University and University of Durham,
	  EBSE Technical Report,
	  2007.

	\setcounter{firstbib}{\value{enumiv}}

\end{thebibliography}

\renewcommand{\refname}{Systematic research review references} 

\begin{thebibliography}{9}  

\setcounter{enumiv}{\value{firstbib}}

 \bibitem{strauch}
  	  Christof Strauch, 
      \emph{NoSQL Databases}. Stuttgart Media University.
      
      %\bibitem{datastax}
  	  %Datastax Corporation, 
      %\emph{Benchmarking Top NoSQL Databases}. February 2013.
      
      \bibitem{survey}
  	  Jing Han, Haihong E, Guan Le,
      \emph{Survey on NoSQL Database}. Beijing University of Posts and Telecommunications.
      
      \bibitem{strauch2}
  	  Christof Strauch, 
      \emph{NoSQL Databases: a step to database scalability in Web environment}. Stuttgart Media University.
      
      \bibitem{pokorny}
  	   Jaroslav Pokorny, 
      \emph{NoSQL Databases}. Charles University.
      
      \bibitem{cattell}
  	  Rick Cattell, 
      \emph{Scalable SQL and NoSQL Data Stores}. Cattell.Net Software.
      
      %\bibitem{xiang}
  	  %Peng Xiang, Ruichun Hou and Zhiming Zhou, 
      %\emph{Cache and Consistency in NOSQL}. Information Engineering Center, Ocean University of China.
      
      \bibitem{leavittm}
  	  Neal Leavittm 
      \emph{Will NoSQL Databases Live Up to Their Promise}. Technology News.
      
      \bibitem{delis}
  	  Alexios Delis, Nick Roussopoulos, 
      \emph{Performance and Scalability of Client-Server Database Architectures}. University of Maryland.
      
      \bibitem{tudorica}
  	  Bogdan George Tudorica, Cristian Bucur, 
      \emph{A comparison between several NoSQL databases with comments and notes}. Petroleum-Gas University of Ploiesti, Ploiesti, Romania.
      
      \bibitem{taniar}
  	  David Taniar, 
      \emph{High Performance Database Processing}. Clayton School of Information Technology, Clayton, Victoria, Australia.
      
      \bibitem{konstantinou}
  	  Ioannis Konstantinou, Evangelos Angelou, Christina Boumpouka, Dimitrios Tsoumakos, Nectarios Koziris, 
      \emph{On the Elasticity of NoSQL Databases over Cloud Management Platforms}. National Technical University of Athens, Greece.
      
      \bibitem{chakrabarti}
  	  Aniket Chakrabarti, Christopher Stewart,
      \emph{Efficent Latency Management in NoSQL Stores}. The Ohio State University.
      
      \bibitem{rizzotti}
  	  Nicolas Ruflin, Helmar Burkhart, Sven Rizzotti, 
      \emph{Social-Data Storage-Systems}.  University of Basel, Switzerland.
      
      \bibitem{tiwari}
  	  Shashank Tiwari, 
      \emph{Professional NoSQL}. Indianapolis, Ind. : John Wiley \& Sons, 2011.
      
      \bibitem{orend}
  	  Kai Orend, 
      \emph{Analysis and Classification of NoSQL Databases and Evaluation of their Ability to Replace an Object-relational Persistence Layer}. Master's Thesis. Technische Universitat Munchen.
      
      \bibitem{prasad}
  	  Rabi Prasad, 
      \emph{RDBMS to NoSQL: Reviewing Some Next-Generation Non-Relational Database's}. Stuttgart Media University.
      
      \bibitem{hecht}
  	  Robin Hecht, Stefan Jablonski,
      \emph{NoSQL Evaluation - A Use Case Oriented Survey}. Patra Berhampur University India.
       
      \bibitem{sakr} 
  	  Sherif Sakr, 
      \emph{Supply cloud-level data scalability with NoSQL databases}. National ICT Australia.
      
      \bibitem{nosql-database}
      \emph{The biggest list of nosql databases} available at
      \url{http://nosql-database.org/}
      
      %\bibitem{nosql-benchmarking}
      %\emph{NoSQL Benchmarking} available at
      %\url{http://www.cubrid.org/blog/dev-platform/nosql-benchmarking/}

\end{thebibliography}
 

%HARMONOGRAM
%Radek: Reseach Method : poniedziałek 8 rano
%Mateusz: Protocol : wtorek 8 rano
%Ireneusz: Execution of protocol czwartek 8 rano
%Radek: Analisis sobota 8 rano
%Mateusz: Conclusion sobota 8 rano
%Irek: Synthesiza sobota 8 rano
%Niedziela wieczor: spotykamy sie i dopinamy calos + initial assigmenet 2 o 17 I MA BYC JUZ PO ANG


%DRUGI
%ASSIGMENT 2 do niedzieli
%BACKGROUND: Ireneusz
%ABSSTRACT + keywords: Mateusz
%introduction: Radek
%Research question: Mateusz
%outcomes: Radek

\end{document}
